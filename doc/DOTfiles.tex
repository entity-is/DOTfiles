%!TEX TS-program = arara
%
% File   : DOTfiles.tex
% Author : ɛntiˈtɛːt.kaɪ̯
% Date   : 2016-03-17
%
%---+----1----+----2----+----3----+----4----+----5----+----6----+----7----+----8
% arara config
%   arara: lualatex
%---+----1----+----2----+----3----+----4----+----5----+----6----+----7----+----8



\documentclass[a4paper,12pt]{article}



	% Pakete
		% Unterstützung für die deutsche Sprache
			\usepackage{polyglossia}

		% TODOs
			\usepackage[obeyFinal,ngerman,colorinlistoftodos]{todonotes}


	% Definitionen
		% Polyglossia
			\setdefaultlanguage[
				variant=german,        % „german“ oder „austrian“
				spelling=new,          % reformierte Trennregeln von 1996
				latesthyphen=true,     % die neusten Trennregeln verwenden
				babelshorthands=true]  % aus „babel“ bekannte Befehle wie „"ck“,
				                       % „"ff“, etc. behalten ihre Bedeutung
				{german}


	% Titelei
		\title{\fontspec{Times New Roman}DOTfiles}
		\author{\fontspec{Times New Roman}ɛntiˈtɛːt.kaɪ̯}
		\date{%
			{\fontspec{Times New Roman}17. März 2016} \\
			{\bf---} \\[0.2\baselineskip]
			\fontspec{Times New Roman}\today}



\begin{document}



	% List of todos
		\listoftodos

	% Titelseite
		\maketitle

	% Zusammenfassung
		%
% File   : zusammenfassung.tex
% Author : ɛntiˈtɛːt.kaɪ̯
% Date   : 2016-03-17
%
%---+----1----+----2----+----3----+----4----+----5----+----6----+----7----+----8



\begin{abstract}
	\todo[inline,caption={Zusammenfassung}]{Zweck des Projekts.}
\end{abstract}



%---+----1----+----2----+----3----+----4----+----5----+----6----+----7----+----8
% vim:wrap:noet:ts=2 sw=2 sts=2


	% Verzeichnisse
		\tableofcontents  % Inhaltsverzeichnis



\end{document}



%---+----1----+----2----+----3----+----4----+----5----+----6----+----7----+----8
% vim:wrap:noet:ts=2 sw=2 sts=2
